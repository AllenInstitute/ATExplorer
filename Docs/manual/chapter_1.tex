\chapterimage{chapterimage_1.jpg} % Chapter heading image

\chapter{Overview of the \ate{} UI and API}
\section{Introduction}\index{Introduction}

\doublespacing
This chapter gives an overview of the software that is named \emph{ATExplorer}.

The following section discusses the application in greater detail.

The \ate was designed and implemented due to an emerging need to allow non-programmers to process, manage and explorer \at data on a routine basis.

Depending on the actual protocols, an \at data set can range in size from pretty small, a few hundred megabytes, to very large, like several Terra bytes and depending on number of stains and sessions, the complexity of the data-sets range from trivial to complex.

The main challenge in \at is the precise reconstruction of an original volume, from individually cut and imaged slices of tissue of the same.

.. even before volume reconstruction can begin various pre data processing algorithms may need to be applied, such as median filtering, flatfield correction and de-convolution.

These processing algorithms are all, to some extent, complex. \ate attempts to provide the non-programmer user with intuitive and easy to use UI components to guide the through the process in order to get to data that are useful for scientific discoveries and exploration.

\clearpage

\section{The \ate UI}

\begin{figure}[h]
\centering\includegraphics[scale=0.85]{ATExplorerUI_1}

\caption{\ate{} UI. The circled numbers in the figure indicate relevant elements of the UI; \protect\cn{1} Project(s) TreeView. \protect\cn{2} Tabbed Project Item View. \protect\cn{3} Information and Application Log Messages.}
\end{figure}

\subsection{Importing Data}

\begin{description}[font=$\bullet$~\normalfont\scshape\color{red!50!black}]
\item [Importing process] Give an overview on what happens when data is being imported to \ate.
\item [Data Formats] Describe the Allen Institute format, and Kristinas format.
\end{description}

\subsection{Processing Data}
\begin{description}[font=$\bullet$~\normalfont\scshape\color{red!50!black}]
\item [Median calculation]
\item [FlatField correction]
\item [Deconvolution]
\item [Stitching]
\item [Registration]
\item [Rough aligning] 
\item [Fine aligning] 
\item [Other]
\end{description}

\begin{figure}[h]
\centering\includegraphics[scale=0.65]{LMDataProcessing_1}

\caption{Processing ..........}
\end{figure}

\subsection{Connecting to a a Remote (or local) RenderHost}

\subsection{Managing Stacks in Render}

\subsection{Exploring Data}


\clearpage


\section{Python Bindings}